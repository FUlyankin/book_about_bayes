% !TEX root = main_file.tex

%%%%%%%%%% Програмный код %%%%%%%%%%
% \usepackage{minted}
% Включает подсветку команд в программах!
% Нужно, чтобы на компе стоял питон, надо поставить пакет Pygments, в котором он сделан, через pip.

% Для Windows: Жмём win+r, вводим cmd, жмём enter. Открывается консоль.
% Прописываем easy_install Pygments
% Заходим в настройки texmaker и там прописываем в PdfLatex:
% pdflatex -shell-escape -synctex=1 -interaction=nonstopmode %.tex

% Для Linux: Открываем консоль. Убеждаемся, что у вас установлен pip командой pip --version
% Если он не установлен, ставим его: sudo apt-get install python-pip
% Ставим пакет sudo pip install Pygments

% Для Mac: Всё то же самое, что на Linux, но через brew.

% После всего этого вы должны почувствовать себя тру-программистами!
% Документация по пакету хорошая. Сам читал, погуглите!

%%%%%%%%%% Математика %%%%%%%%%%
\usepackage{amsmath,amsfonts,amssymb,amsthm,mathtools}

%%%%%%%%%%%%%%%%%%%%%%%% Шрифты %%%%%%%%%%%%%%%%%%%%%%%%%%%%%%%%%
\usepackage[english, russian]{babel}  % выбор языка для документа
\usepackage[utf8]{inputenc}           % задание utf8 кодировки исходного tex файла
\usepackage[X2,T2A]{fontenc}          % кодировка

% Основные шрифты 
\usepackage{fontspec}         
\setmainfont{Linux Libertine O}  % задаёт основной шрифт документа

% Математические шрифты 
\usepackage{unicode-math}     
\setmathfont[math-style=upright]{[Neo Euler.otf]} 

% Конкретный символ из конкретного шрифта
% \setmathfont[range=\int]{Neo Euler}


%%%%%%%%%% Работа с картинками %%%%%%%%%
\usepackage{graphicx}                  % Для вставки рисунков
\usepackage{graphics}
\graphicspath{{images/}{pictures/}}    % можно указать папки с картинками
\usepackage{wrapfig}                   % Обтекание рисунков и таблиц текстом
\usepackage[figurename=Картинка]{caption}


%%%%%%%%%% Работа с таблицами %%%%%%%%%%
\usepackage{tabularx}            % новые типы колонок
\usepackage{tabulary}            % и ещё новые типы колонок
\usepackage{array}               % Дополнительная работа с таблицами
\usepackage{longtable}           % Длинные таблицы
\usepackage{multirow}            % Слияние строк в таблице
\usepackage{float}               % возможность позиционировать объекты в нужном месте
\usepackage{booktabs}            % таблицы как в книгах!
\renewcommand{\arraystretch}{1.2} % больше расстояние между строками

% Заповеди из документации к booktabs:
% 1. Будь проще! Глазам должно быть комфортно
% 2. Не используйте вертикальные линни
% 3. Не используйте двойные линии. Как правило, достаточно трёх горизонтальных линий
% 4. Единицы измерения - в шапку таблицы
% 5. Не сокращайте .1 вместо 0.1
% 6. Повторяющееся значение повторяйте, а не говорите "то же"
% 7. Есть сомнения? Выравнивай по левому краю!

%%%%%%%%%% Графика и рисование %%%%%%%%%%
\usepackage{tikz, pgfplots}  % языки для графики
%\pgfplotsset{compat=1.16}


%%%%%%%%%% Другие приятные пакеты %%%%%%%%%
\usepackage{multicol}       % несколько колонок
\usepackage{verbatim}       % для многострочных комментариев
\usepackage{cmap}           % для кодировки шрифтов в pdf

% свешиваем пунктуацию
% теперь знаки пунктуации могут вылезать за правую границу текста, при этом текст выглядит ровнее
\usepackage{microtype}
\usepackage{epigraph} % для эпиграфов

\usepackage{enumitem} % дополнительные плюшки для списков
%  например \begin{enumerate}[resume] позволяет продолжить нумерацию в новом списке

\usepackage{todonotes} % для вставки в документ заметок о том, что осталось сделать
% \todo{Здесь надо коэффициенты исправить}
% \missingfigure{Здесь будет Последний день Помпеи}
% \listoftodos --- печатает все поставленные \todo'шки

%%%% Оформление %%%%%%%
\usepackage[
paperwidth=160mm,
paperheight=220mm,
headheight=14mm,
left=10mm,
right=10mm,
top=20mm,
bottom=20mm
]{geometry}

\usepackage{indentfirst}    % установка отступа в первом абзаце главы


\usepackage{setspace}
%\setstretch{1.3}  % Межстрочный интервал
%\setlength{\parindent}{1.5em} % Красная строка.
%\setlength{\parskip}{4mm}   % Расстояние между абзацами
% Разные длины в латехе https://en.wikibooks.org/wiki/LaTeX/Lengths

% \flushbottom                            % Эта команда заставляет LaTeX чуть растягивать строки, чтобы получить идеально прямоугольную страницу
\righthyphenmin=2                       % Разрешение переноса двух и более символов
\widowpenalty=300                     % Небольшое наказание за вдовствующую строку (одна строка абзаца на этой странице, остальное --- на следующей)
\clubpenalty=3000                     % Приличное наказание за сиротствующую строку (омерзительно висящая одинокая строка в начале страницы)
\tolerance=10000     % Ещё какое-то наказание.

\usepackage{bm}
\usepackage{bbm} % шрифт с двойными буквами

% свешиваем пунктуацию
% теперь знаки пунктуации могут вылезать за правую границу текста, при этом текст выглядит ровнее
\usepackage{microtype}

% для эпиграфов
\usepackage{epigraph} 
\setlength\epigraphrule{0pt}
\setlength{\epigraphwidth}{0.5\textwidth}
\renewcommand{\textflush}{flushepinormal}

% Внешний вид подписей к картинкам и таблицам
\usepackage[font=small, labelfont=bf]{caption}
\DeclareCaptionLabelSeparator{colon}{\textbf{.} }
\DeclareCaptionLabelFormat{dash}{#1\hspace{.55ex}#2}
\captionsetup[figure]{labelformat=dash}


% мои цвета https://www.artlebedev.ru/colors/
\definecolor{titleblue}{rgb}{0.2,0.4,0.6} 
\definecolor{blue}{rgb}{0.2,0.4,0.6} 
\definecolor{red}{rgb}{1,0,0.2} 
\definecolor{green}{rgb}{0,0.6,0} 
\definecolor{purp}{rgb}{0.4,0,0.8} 

% цвета из geogebra 
\definecolor{litebrown}{rgb}{0.6,0.2,0}
\definecolor{darkbrown}{rgb}{0.75,0.75,0.75}


% Гиперссылки
\usepackage{xcolor}   % разные цвета

\usepackage{hyperref}
\hypersetup{
	unicode=true,           % позволяет использовать юникодные символы
	colorlinks=true,       	% true - цветные ссылки
	urlcolor=blue,          % цвет ссылки на url
	linkcolor=black,          % внутренние ссылки
	citecolor=green,        % на библиографию
	breaklinks              % если ссылка не умещается в одну строку, разбивать её на две части?
}

% меняю оформление секций 
\usepackage{titlesec}
\usepackage{sectsty}

% меняю цвет на синий
\sectionfont{\color{titleblue}}
\subsectionfont{\color{titleblue}}

% синие круглые бульпоинты в списках itemize 
\usepackage{enumitem}

\definecolor{itemizeblue}{rgb}{0, 0.45, 0.70}

\newcommand*{\MyPoint}{\tikz \draw [baseline, fill=itemizeblue, draw=blue] circle (2.5pt);}
\renewcommand{\labelitemi}{\MyPoint}

\AddEnumerateCounter{\asbuk}{\@asbuk}{\cyrm}
\renewcommand{\theenumi}{\asbuk{enumi}}

% расстояние в списках
\setlist[itemize]{parsep=0.4em,itemsep=0em,topsep=0ex}
\setlist[enumerate]{parsep=0.4em,itemsep=0em,topsep=0ex}


%%%%%%%%%% Свои команды %%%%%%%%%%
\usepackage{etoolbox}    % логические операторы для своих макросов

%%%%%%%%%% Теоремы %%%%%%%%%%
\theoremstyle{plain}              % Это стиль по умолчанию.  Есть другие стили.
\newtheorem{theorem}{Теорема}[section]
\newtheorem{result}{Следствие}[theorem]
% счётчик подчиняется теоремному, нумерация идёт по главам согласованно между собой


\theoremstyle{definition}         % убирает курсив и что-то еще наверное делает ;)
\newtheorem*{definition}{Определение}  % нумерация не идёт вообще

\newtheorem{chudo}{Чудо номер}   % Для первой главы


%%%%%%%%%% Задачи и их решения %%%%%%%%%%%
\usepackage{answers}

\newtheorem{problem}{\color{myblue} Упражнение}
\Newassociation{sol}{solution}{solution_file}
% sol --- имя окружения внутри задач
% solution --- имя окружения внутри solution_file
% solution_file --- имя файла в который будет идти запись решений
% можно изменить далее по ходу


\usepackage{pgf,tikz}
\usepackage{mathrsfs}
\usetikzlibrary{arrows}


% \usepackage{environ}
% \newtoggle{lecture}

% \newcounter{probNum}[section]  % счётчик для упражнений 
% \NewEnviron{problem}[1]{%
%     \refstepcounter{probNum}% увеличели номер на 1 
%     {\noindent \textbf{\large \color{titleblue} Упражнение~\theprobNum~#1}  \\ \\ \BODY}
%     {}%
%   }

% % Окружение, чтобы можно было убирать решения из pdf
% \NewEnviron{sol}{%
%   \iftoggle{lecture}
%     {\noindent \textbf{\large Решение:} \\ \\ \BODY}
%     {}%
%   }


% разные дополнения для картинок
\usetikzlibrary{arrows.meta}
\usepackage{varwidth}

% рисование крестов
% https://tex.stackexchange.com/questions/123760/draw-crosses-in-tikz
\tikzset{
    cross/.pic = {
    \draw[line width=2.pt, rotate = 45] (-#1,0) -- (#1,0);
    \draw[line width=2.pt, rotate = 45] (0,-#1) -- (0, #1);
    }
}
   
\usepackage[normalem]{ulem}  % для зачекивания текста
